\begin{table}[h]
\centering
\begin{tabular}{|cc|c|c|c|}
\hline
\multicolumn{1}{|c|}{\textbf{Nr}} & \textbf{Division}                 & \textbf{Population}  & \textbf{Area}    & \textbf{Density} \\ \hline
\multicolumn{1}{|c|}{\textbf{1}}  & {\color[HTML]{000000} Barisal}    & 8 147 000            & 13 297           & 613              \\ \hline
\multicolumn{1}{|c|}{\textbf{2}}  & {\color[HTML]{000000} Chittagong} & 28 079 000           & 33 771           & 831              \\ \hline
\multicolumn{1}{|c|}{\textbf{3}}  & {\color[HTML]{000000} Dhaka}      & 46 729 000           & 31 120           & 1502             \\ \hline
\multicolumn{1}{|c|}{\textbf{4}}  & {\color[HTML]{000000} Khulna}     & 15 563 000           & 22 272           & 699              \\ \hline
\multicolumn{1}{|c|}{\textbf{5}}  & {\color[HTML]{000000} Rajshahi}   & 18 329 000           & 18 197           & 1007             \\ \hline
\multicolumn{1}{|c|}{\textbf{6}}  & {\color[HTML]{000000} Rangpur}    & 15 665 000           & 16 317           & 960              \\ \hline
\multicolumn{1}{|c|}{\textbf{7}}  & {\color[HTML]{000000} Sylhet}     & 9 807 000            & 12 596           & 779              \\ \hline
\multicolumn{2}{|c|}{\textbf{Bangladesh}}                             & \textbf{142 319 000} & \textbf{147 570} & \textbf{964}     \\ \hline
\end{tabular}
\caption{Administrative Divisions of Bangladesh}
\label{tab:bang}
\end{table}